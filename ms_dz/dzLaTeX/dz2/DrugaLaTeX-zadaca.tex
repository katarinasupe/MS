\documentclass[10pt]{scrartcl}
\usepackage[utf8]{inputenc}
\usepackage[croatian]{babel}
\usepackage{csquotes}
\MakeOuterQuote{"}
\usepackage{amsmath,amssymb,amsthm}
\usepackage[unicode]{hyperref}
\usepackage{thmtools}
\declaretheorem{teorem}
\declaretheorem[style=definition,sibling=teorem]{definicija}
\declaretheorem[style=remark,qed=$\|$,sibling=teorem]{primjer}
\declaretheorem[style=remark,sibling=teorem]{napomena}
\declaretheorem[style=definition]{zadatak}
\newcommand{\D}{\,\mathrm d}
\usepackage{marvosym}

\begin{document}
\title{Druga \LaTeX-zadaća}
\author{Katarina Šupe}
\date{Zagreb, \today}
\maketitle
\tableofcontents
\section{Limes niza u \texorpdfstring{$\mathbb R$}{R}}
\begin{definicija}\label{tm:convergence}
Niz realnih brojeva $(a_n)_n$ \emph{konvergira} ili teži k realnom broju $a \in \mathbb R$ ako za svaki otvoreni interval polumjera $\varepsilon$ oko točke $a$ sadrži gotovo sve članove niza, tj.
\begin{equation}\label{eq:limit}
    (\forall \varepsilon > 0)(\exists n_{\varepsilon} \in \mathbb N)(\forall n \in \mathbb N) \left( ( n > n_{\varepsilon} ) \Rightarrow (|a_n -a| < \varepsilon ) \right) \tag{\Coffeecup}
\end{equation}
Tada $a$ zovemo \emph{granična vrijednost} ili \emph{limes} niza $(a_n)_n$ i pišemo $a = \lim_{n \to \infty} a_n$ ili $a = \lim_n a_n$. Ako niz ne konvergira, onda kažemo da \emph{divergira}.
\end{definicija}
\begin{teorem} 
Za konvergentan niz vrijede sljedeće tvrdnje:
\begin{enumerate}
    \item Konvergentan niz u $\mathbb R$ ima samo jednu graničnu vrijednost.
    \item Konvergentan niz u $\mathbb R$ je ograničen.
\end{enumerate}
\end{teorem}
\begin{proof}
\begin{enumerate}
\item Pretpostavimo da konvergentan niz $(a_n)_n$ ima dvije granične vrije\-dno\-sti $a,b \in \mathbb R$, $a \neq b$. Tada bi za $\varepsilon = |a-b|>0$ postojali $n_a,n_b \in \mathbb N$ takvi da vrijedi
\begin{align*}
(n>n_a) &\Rightarrow \left(|a_n - a| < \frac{\varepsilon}{2} \right) \text{ i}\\
(n>n_b) &\Rightarrow \left(|a_n - b| < \frac{\varepsilon}{2} \right)
\end{align*}
Sada za $n_{\varepsilon} = \max\{n_a, n_b\}$ imamo
\[
(n>n_\varepsilon) \Rightarrow \left(|a-b| \leq |a-a_n| + |a_n - b| < \frac{\varepsilon}{2} + \frac{\varepsilon}{2} = \varepsilon = |a-b| \right)
\]
što je očita neistina. Dakle, limes mora biti jedinstven. 
\item U formuli~\eqref{eq:limit} uzmimo $\varepsilon = 1$, pa postoji $n_{\varepsilon} \in \mathbb N$ tako da $(n>n_{\varepsilon}) \Rightarrow (|a_n - a| <1)$. Sada za $n>n_{\varepsilon}$ imamo 
\[
|a_n| \leq |a_n - a| + |a| \leq 1 + |a|
\]
Neka je $M = \max\{|a_1|,\dotsc, |a_{n_{\varepsilon}}|, 1 + |a| \}$. Tada vrijedi 
\[
|a_n|\leq M,\text{ za sve } n \in \mathbb N,
\]
tj. niz je ograničen.
\end{enumerate}
\end{proof}
%
\section{Derivacija}
\begin{definicija}\label{tm:derivation}
Kažemo da je funkcija  $f\colon I\to \mathbb R$ diferencijabilna ili derivabilna u točki $c$ otvorenog intervala $I \subseteq \mathbb R$, ako postoji $\lim_{x \to c} \frac{f(x) - f(c)}{x-c}$. Taj broj zovemo \emph{derivacija} (izvod) funkcije $f$ u točki $c$ i pišemo 
\begin{equation}
f'(c) = \lim_{x \to c} \frac{f(x) - f(c)}{x-c}
\end{equation}
\end{definicija}
\begin{primjer}
Koristeći definiciju~\ref{tm:derivation} nađi derivaciju konstantne funkcije $f(x) = \alpha$, $\forall x \in \mathbb R$ u točki $c \in \mathbb R$.
Vrijedi
\[
f'(c) = \lim_{x \to c} \frac{f(x) - f(c)}{x-c} = \lim_{x\to c} \frac{\alpha - \alpha}{x-c} = 0
\]
$\leadsto$ funkcija $f$ ima derivaciju $0$ u svim realnim brojevima.
\end{primjer}
%
\section{Riemannov Integral}
\begin{definicija}
Broj $\mathcal{I}_*$ zovemo \emph{donji Riemannov integral} funckije $f$ na segmentu $[a,b]$, a broj $\mathcal{I}^*$ zovemo \emph{gornji Riemannov integral} funkcije $f$ na segmentu $[a,b]$. 
\end{definicija}
\begin{definicija}
Za funkciju $f \colon [a,b] \to \mathbb R$ ograničenu na segmentu $[a,b]$ kažemo da je \emph{integrabilna u Riemannovom smislu} ili \emph{R-integrabilna} na segmentu $[a,b]$ ako je
\begin{equation}
\mathcal{I}_*(f;a,b) = \mathcal{I}^*(f;a,b) 
\end{equation}
Tada se broj $\mathcal{I} = \mathcal{I}_* = \mathcal{I}^* $ naziva \emph{integral} ili \emph{R-integral} funkcije $f$ na segmentu $[a,b]$ i označava jednom od sljedećih oznaka
\begin{equation}
\mathcal{I} = \int_{[a,b]} f(t)\D t = \int_a^b\!f(x)\,\D x = \int_{[a,b]} f = \int\limits_a^b\,f
\end{equation}
\end{definicija}
\section{Kvadratna funkcija}
\begin{definicija}
Neka su $a,b,c \in \mathbb R$, $a \neq 0$. \emph{Kvadratna funkcija} je funkcija $f \colon \mathbb R \to \mathbb R$ zadana formulom $f(x) = ax^2 + bx + c$. 
\end{definicija}
Nadopunjavanjem do potpunog kvadrata dolazimo do formule koja određuje nul\-to\-čke kvadratne funkcije
\begin{equation}
x_{1,2} = \frac{-b \pm \sqrt{b^2 - 4ac}}{2a}
\end{equation}
\begin{napomena}
Kvadratna jednadžba se može shvatiti kao poseban slučaj kvadratne funkcije $y = f(x)$ za vrijednost funkcije $y=0$, gdje tada rješenja kvadratne jednadžbe predstavljaju nutočke kvadratne funkcije. 
\end{napomena}
%
\section{Domaća zadaća}
\begin{zadatak}
Matematičkom indukcijom dokažite
\[
1+2+2^2+\dotsb+2^n = 2^{n+1} -1
\]
za svaki prirodan broj $n$. Nadalje, dokažite da za dani $x \neq 1$ vrijedi
\[
1+x+x^2+x+\dotsb+x^n = \frac{x^{n+1} -1}{x-1}
\]
\end{zadatak}
\begin{zadatak}
Neka su $a,b,c,d,k,l \in \mathbb Z$, te neka je $p$ prost broj. Pokažite da sustav kongruencija ima jedinstveno rješenje $(x,y)\in\{1,2,\dotsc,p\}\times\{1,2,\dotsc,p\}$ ako i samo ako za determinantu sustava vrijedi $\left|\begin{array}{cc} 
a & b \\ 
c & d 
\end{array}\right| \not \equiv 0\,(\text{mod } p)$.
\end{zadatak}
\section{Literatura}
\url{https://web.math.pmf.unizg.hr/~guljas/skripte/MATANALuR.pdf}\\
\url{https://web.math.pmf.unizg.hr/nastava/em/EM1/kolokviji/EM1-kol2.pdf}\\
\url{https://web.math.pmf.unizg.hr/~gmuic/predavanja/uum.pdf}\\
\url{https://hr.wikipedia.org/wiki/Kvadratna_jednad%C5%BEba}
\end{document}

