\documentclass[12pt]{scrartcl}
\usepackage[utf8]{inputenc}
\usepackage[croatian]{babel}
\usepackage{csquotes}
\MakeOuterQuote{"}
\usepackage{amsmath,amssymb,amsthm}
\usepackage[unicode]{hyperref}
\usepackage{thmtools}
\declaretheorem{teorem}
\declaretheorem[style=definition]{zadatak}
\usepackage{graphicx}
\newcommand{\norm}[1]{\left\|#1\right\|}
% w' mi se ljepše renderira s dodanim razmakom, pa sam napravila naredbu takvu da ne moram dodavati razmak svaki put kad pišem w'
\newcommand{\tick}[1]{#1\,'}


\begin{document}
\title{Treća LaTeX-zadaća}
\author{Katarina Šupe}
\date{Zagreb, \today}
\maketitle
\tableofcontents
\section{Teorem, matrica, tablica i Sunce}
\begin{teorem}[o najboljoj aproksimaciji]\label{tm:aproksimacija}
Neka je $W$ potprostor unitarnog prostora $V$ te neka je $v = w + u$, gdje su $w\in W$ te $u \in W^{\perp}$. Tada vrijedi: 
\[
\norm{v-w} < \norm{v-\tick{w}}
\]
za svaki $\tick{w}\in W$ takav da je $\tick{w} \neq w$.
\end{teorem}
\begin{proof}
Po definiciji ortogonalne projekcije imamo $v-w \perp W$. Kako je $w-\tick{w} \in W$, to je $v-w \perp w-\tick{w}$. Također, za $\tick{w} \neq w$ imamo $\norm{w-\tick{w}}>0$. Stoga možemo primijeniti Pitagorin poučak:
\[
\norm{v-\tick{w}}^2 = \norm{(v-w) + (w-\tick{w})}^2 = \norm{v-w}^2 + \norm{w-\tick{w}}^2 > \norm{v-w}^2\text.\qedhere
\]
\end{proof}

Glavni \textsl{goal} nam je napisati riječ na engleskom jeziku u drukčijem fontu. Osim toga, \emph{ne zaboravimo naglasiti dio teksta}. Povrh svega, najvažnije je da je skup $\mathbb{N}$ skup prirodnih brojeva. 

\begin{equation}\label{eq:Vandermonde}
V=\begin{bmatrix}
1 & \alpha_1 & {\alpha_1}^2 & \cdots & {\alpha_1}^{n-1} \\
1 & \alpha_2 & {\alpha_2}^2 & \cdots & {\alpha_2}^{n-1} \\
1 & \alpha_3 & {\alpha_3}^2 & \cdots & {\alpha_3}^{n-1} \\
\vdots &\vdots & \vdots & \ddots & \vdots \\
1 & \alpha_m & {\alpha_m}^2 & \cdots & {\alpha_m}^{n-1} \\
\end{bmatrix}
\end{equation}
Matrica~\eqref{eq:Vandermonde} je Vandermondeova matrica.
\begin{table}[ht]
\centering
\begin{tabular}{llll}
   A & B & C & D \\ \hline
\multicolumn{3}{l}{tri stupca} & 1\\
   2 & 3 & 4 & 5 \\
   6 & 7 & 8 & 9
\end{tabular}
\end{table}

\begin{figure}[ht]
\caption{Sunce}
\begin{center}
\includegraphics[scale=0.1]{sunce}
\end{center}
\label{fig:sunce} 
\end{figure}

\section{Biološka klasifikacija}
\begin{itemize}
\item[Vrsta] je osnovna jedinica biološke raznovrsnosti.
\item[Rod] je grupa vrsta živih bića, fosilnih organizama i virusa.
\item[Porodica] predstavlja taksonomski rang između reda i roda. Porodica obično uključuje više rodova, a više porodica čini red.
\item[Red] predstavlja taksonomsku jedinicu koja se nalazi između razreda i porodice.
\item[Razred] se nalazi između koljena i reda.
\item[Koljeno] se nalazi između carstva i razreda.
\item[Carstvo] je dugo vremena bilo najviša kategorija razvrstavanja živih organizama. 
\item[Domena] je najviša kategorija klasificiranja živih bića (\textsl{Carl R. Woese}).
\end{itemize}

Većina hijerarhijskih stepenica dalje se dijeli:
\begin{enumerate}
\item Razed
    \begin{enumerate}
        \item podrazred
    \end{enumerate}
\item Koljeno
\begin{enumerate}
    \item podkoljeno
\end{enumerate}
\item [$\vdots$]
\end{enumerate}

\section{\LaTeX{} provjera}
\begin{zadatak}
Pokušajte rekreirati tablicu: 
\end{zadatak}
\begin{table}[h]
\begin{tabular}{|l|l|l|l|}
\hline
\multicolumn{4}{|l|}{$1+x+\dotsb+x^n$}\\ \hline
\multicolumn{2}{|l|}{$\dotsc \vdots$} & 
\multicolumn{2}{l|}{$\xi = \bigstar$}\\ \hline
$\mathbb N/_\sim$ & \multicolumn{3}{l|}{\emph{jako važno!}} \\ \hline
\end{tabular}
\end{table}

\section{Literatura}
\url{https://hr.wikipedia.org/wiki/Sistematika}\\
\url{https://web.math.pmf.unizg.hr/~gmuic/predavanja/vp.pdf}
\end{document}

